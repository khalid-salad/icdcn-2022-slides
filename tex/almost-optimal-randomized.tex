\section{An Almost-Optimal Randomized Algorithm}
\begin{frame}{Sketch-Based Randomized Algorithm}
    \begin{itemize}
        \item we analyze the 2018 SPAA algorithm of Pandurangan, Robinson, and Scquizzato
        \item optimal up to polylog factors in terms of round complexity
        \begin{itemize}
            \item but $\softO{n^2}$ local computation complexity
        \end{itemize}
        \item the algorithm is similar to Bor\r{u}vka's algorithm, with
              the output being a \alert{labeling} of the nodes, such that nodes in the same component have the same label
        % \item component also have a \alert{component proxy} that handles finding MOEs and merging for the component
        % \item initially, each node has label equal to its ID (forming its own component) and is its own component proxy
        % \item to load balance communication and computation, component proxies are chosen randomly among the machines
        % \item given a component $f$ and a machine $M$, the set of nodes belonging
        %       to $f$ in $M$ are called the \alert{component part}
        \item relies on \alert{linear graph sketches} to efficiently merge multiple components
    \end{itemize}
\end{frame}

\begin{frame}{Sketches}
    \begin{itemize}
        \item sketches are found by forming a vector of length
              $\binom{n}{2}$ for each node $v$ and and appropriate choice of an
              $\bigO{\log^2{n}} \times \binom{n}{2}$ matrix. This requires $\softOm{n^2}$ local computation time.
        \item can be improved by using the 2015 sketches of King, Kutten, and Thorup
        \item instead of storing vectors of length $\binom{n}{2}$, a leader machine broadcasts an \alert{odd hash function}
        \item each component uses this hash function to find an MOE
        \item brings it to $\lcc = \softO{\sfrac{(m+n)}{k} + \Delta + k}$
        \begin{itemize}
            \item optimal, up to polylog factors!
        \end{itemize}
    \end{itemize}
\end{frame}

% \begin{frame}
%     \begin{itemize}
%         \item each component uses this hash function to find an MOE as follows
%               \begin{itemize}
%                   \item each node in a component part evaluates $\sum_{\text{incident edges } e} h(e)
%                             \bmod{2}$
%                   \item this sum is aggregated among the nodes in a component part by the
%                         corresponding machine
%                   \item the sums of each part of the component are aggregated to the
%                         corresponding component proxy machine.
%                   \item If the sum is 1, we conclude with some constant probability
%                         $\varepsilon$ that an outgoing edge exists out of the component
%                         \begin{itemize}
%                             \item By restricting the edges considered to be those within some range $[i, j]$, can (like binary search) determine an MOE with constant probability
%                             \item repeating this process $\bigOm{\log{n}}$ times yields an
%                                   MOE whp for a single component
%                             \item by union bound, we have MOE for all components whp
%                         \end{itemize}
%               \end{itemize}
%     \end{itemize}
% \end{frame}