\section{Introduction}
\begin{frame}{The $k$-Machine Model}
    \begin{itemize}
        \item study distributed algorithms for large-scale graphs
        \item focus on connectivity and MST
        \item in $k$-machine model
              \begin{itemize}
                  \item $k \geq 2$ machines jointly perform computations on input graph
                  \item input graph distributed uniformly at random to $k$ machines
                        \begin{itemize}
                            \item $n$ nodes, $m$ edges
                            \item when a node is given to a machine, its \alert{adjacency list} is also
                            \item commonly called \alert{vertex centric} model
                        \end{itemize}
                  \item assume $n \gg k$ (e.g. $k = \bigO{n^{\varepsilon}}$)
              \end{itemize}
    \end{itemize}
\end{frame}


\begin{frame}
    \onslide<1-4>{
    \begin{tikzpicture}[every node/.style={draw}]
        \onslide<1-2>{
        \node[circle,label=above:{Input Graph}] (a) {$a$};
        \node[circle,below left=of a]  (b)  {$b$};
        \node[circle,below=of a] (e) {$e$};
        \node[circle,below right=of a] (f) {$f$};
        \node[circle,below=of b] (c) {$c$};
        \node[circle,below right=of c] (d) {$d$};
        \node[circle,right=of d] (h) {$h$};
        \node[circle,above=of h] (g) {$g$};

        \draw (a) -- (b) -- (c) -- (d) -- (h) -- (g) -- (f) -- (a);
        \draw (a) -- (e) -- (h);
        }

        \node[rectangle, minimum width=2.5cm, minimum height=2cm,label=above:{$M_1$},purple,very thick, above right=of g] (m1) {};
        \node[rectangle, minimum width=2.5cm, minimum height=2cm,label=above:{$M_2$},purple,very thick,right=of m1] (m2) {};
        \node[rectangle, minimum width=2.5cm, minimum height=2cm,label=below:{$M_3$},purple,very thick,below=of m1] (m3) {};
        \node[rectangle, minimum width=2.5cm, minimum height=2cm,label=below:{$M_4$},purple,very thick,right=of m3] (m4) {};

        \draw[<->, very thick, purple] (m1) -- (m2);
        \draw[<->, very thick, purple] (m1) -- (m3);
        \draw[<->, very thick, purple] (m1.south east) -- (m4.north west);
        \draw[<->, very thick, purple] (m2) -- (m4);
        \draw[<->, very thick, purple] (m2.south west) -- (m3.north east);
        \draw[<->, very thick, purple] (m3) -- (m4);

        \onslide<2>{
            \draw[->, dashed] (a) to[bend left] (m1);
            \draw[->, dashed] (e) to[bend left] (m1);
            \draw[->, dashed] (c) to[bend right] (m2);
            \draw[->, dashed] (h) to[bend left] (m2);
            \draw[->, dashed] (b) to[bend right] (m3);
            \draw[->, dashed] (g) to[bend right] (m3);
            \draw[->, dashed] (d) to[bend right] (m4);
            \draw[->, dashed] (f) to[bend right] (m4);
        }

        \onslide<3,4>{
        \node[circle] (a1) at ([xshift=-0.5cm,yshift=0.5cm]m1.center) {$a$};
        \node[circle] (e1) at ([xshift=0.5cm,yshift=-0.5cm]m1.center) {$e$};
        \node[circle] (c1) at ([xshift=-0.5cm,yshift=0.5cm]m2.center) {$c$};
        \node[circle] (h1) at ([xshift=0.5cm,yshift=-0.5cm]m2.center) {$h$};
        \node[circle] (b1) at ([xshift=-0.5cm,yshift=0.5cm]m3.center) {$b$};
        \node[circle] (g1) at ([xshift=0.5cm,yshift=-0.5cm]m3.center) {$g$};
        \node[circle] (d1) at ([xshift=-0.5cm,yshift=0.5cm]m4.center) {$d$};
        \node[circle] (f1) at ([xshift=0.5cm,yshift=-0.5cm]m4.center) {$f$};
        }

        \onslide<4>{
        \draw[dashed, red] (a1) -- (b1) -- (c1) -- (d1) -- (h1) -- (g1) -- (f1) -- (a1);
        \draw[dashed, red] (a1) -- (e1) -- (h1);
        }
    \end{tikzpicture}
    }
\end{frame}

\begin{frame}
    \begin{itemize}
        \item typically, \alert{only consider communication cost}
              \begin{itemize}
                  \item message complexity --- number of communication rounds
              \end{itemize}
        \item traditionally done because network communication is much slower
              than local computation
              \begin{itemize}
                  \item less true as network speeds increase
              \end{itemize}
    \end{itemize}
\end{frame}


\begin{frame}{Our Results}
    \begin{itemize}
        \item we posit a new complexity measure
              \begin{itemize}
                  \item \alert{local computation cost} --- $\lcc$
                  \item measures the worst-case local computation cost among $k$ machines
              \end{itemize}
        \item we refer to traditional communication complexity by $\mcc$
        \item lower bounds
              \begin{align*}
                  \lcc & = \bigOm{\frac{m + n}{k} + \Delta + k} \\
                  \mcc & = \bigOm{\frac{n}{k^2}}
              \end{align*}
        \item in practical implementations of $k$-machine model algorithms,
              often see suboptimal speedup
              \begin{itemize}
                  \item e.g. PageRank with $T_c = \bigO{\sfrac{n}{k}}$ but whose wall-clock
                        time is approximately $\sfrac{n}{k^{0.8}}$
              \end{itemize}
    \end{itemize}
\end{frame}

\begin{frame}
    \begin{itemize}
        \item analyze several algorithms under the new complexity measure
              \begin{tabular}{@{}lll@{}}\toprule
                  Algorithm                  & Round complexity          & Local runtime                        \\\midrule
                  Flooding                   & $\softO{\frac{n}{k} + D}$ & $\softO{\frac{m}{k} + \Delta + k}$   \\
                  Filtering                  & $\softO{\frac{n}{k}}$     & $\softO{\frac{m}{k} + n}$            \\
                  Improved Local Bor\r{u}vka & $\softO{\frac{n}{k}}$     & $\softO{\frac{m+n}{k} + \Delta + k}$ \\
                  Randomized CC              & $\softO{\frac{n}{k^2}}$   & $\softO{\frac{m+n}{k} + \Delta + k}$
              \end{tabular}
        \item many distributed algorithms that are optimal are often only optimal under \alert{communication complexity}
        \item as a byproduct of our analysis, we have two general results
              \begin{itemize}
                  \item the Node Distribution Lemma
                  \item the Mapping Lemma
              \end{itemize}
    \end{itemize}
\end{frame}

\begin{frame}
    \begin{lemma}[Node Distribution Lemma]
        Consider a graph $G$ of nodes $v_1, v_2, \hdots, v_n$ with associated
        non-negative real-valued ``weights'' $w(v)$ for each node $v$. Given a
        uniform, random distribution of the $n$ nodes to $k$ machines, as in the
        $k$-machine model, then, with probability at least $1 - \sfrac{1}{n^a}$ for any
        $a > 0$, the total weight of nodes at every machine is bounded above by
        \[\bigO{T_{\text{avg}}+\log{n}\cdot w_{\text{max}}}\]
        where $T_{\text{avg}}=\frac{1}{k}\sum_{i=1}^n w(v_i)$ and
        $w_{\text{max}} = \text{max}\set{w(v_i)}$.
    \end{lemma}
\end{frame}

\begin{frame}
    \begin{lemma}[Mapping Lemma]\label{lem:more-acc-mapping-lemma}
        Let an \(n\)-node, \(m\)-edge graph \(G\) be partitioned among the \(k\)
        machines as \(N = \set{p_1, \ldots, p_k}\). Then with probability at
        least \(1 - \sfrac{1}{n^\alpha}\), where \(\alpha > 1\) is an arbitrary
        fixed constant, the following bounds hold:
        \begin{enumerate}
            \item The number of vertices mapped to any machine is \(\bigO{n/k}\).
            \item The number of edges mapped to any machine is \(\mathcal{O}(m/k + \Delta \log n)\).
            \item The number of edges mapped to any link of the network is \(\bigO{m/k^2 + n/k}\).
        \end{enumerate}
    \end{lemma}
\end{frame}